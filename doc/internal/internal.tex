\documentclass{article}

\usepackage{fullpage}
\usepackage{hyperref}
\usepackage{color}
\usepackage{listings}

\lstset{tabsize=2}
\lstset{basicstyle=\footnotesize\ttfamily}
\lstset{language=C++}
\lstset{frame=shadowbox, rulesepcolor=\color{blue}}

\begin{document}

\tableofcontents

\newpage

\section{Bugs}

\section{Coding Standards}

This project follows \href{http://google-styleguide.googlecode.com/svn/trunk/cppguide.xml}{Google Coding Standard} with several extensions/exception:

\begin{itemize}
\item Functions do not use CamelCase! Use Unix like naming convention.

\item Never put comma at the beginning of a new line. It significantly reduces code readability especially with 2 space tabs. For instance: 

\begin{lstlisting}
typedef boost::fusion::result_of::make_map<
  boost::mpl::int_<1>, 
  boost::mpl::int_<2>,
  ...

typedef boost::fusion::result_of::make_map<
  boost::mpl::int_<1>
  , boost::mpl::int_<2>
  ...
\end{lstlisting}

\item Using-declarations are only allowed in function definitions.

\item In complicated segments of code (i. e. meta-functions), it is an imperative to make code readable as much as possible. This is usually achieved by using using-directive and proper indentation. For instance consider this improper example: 

\begin{lstlisting}

boost::function<void (content_ptr_t)> _deposit_chunk_callback = &_deposit_chunk;
boost::function<void (content_ptr_t)> _retrieve_chunk_callback = &_retrieve_chunk;
boost::function<void (content_ptr_t)> _heartbeat_callback = &_heartbeat;

boost::fusion::at_key< boost::mpl::int_<DEPOSIT_CHUNK> >(callback_map) =  _deposit_chunk_callback;
boost::fusion::at_key< boost::mpl::int_<RETRIEVE_CHUNK> >(callback_map) =  _retrieve_chunk_callback;
boost::fusion::at_key< boost::mpl::int_<HEARTBEAT> >(callback_map) =  _heartbeat_callback;

uledfs::intercom::masternode client(io_service, callback_map);
client.handle_init(hostname, DEPOSIT_CHUNK);

\end{lstlisting}

Following the suggestion above, this code should look like:

\begin{lstlisting}
{ using boost::mpl::int_;
  using boost::function;
  using boost::fusion::at_key;

function<void (content_ptr_t)> _deposit_chunk_callback  = &_deposit_chunk;
function<void (content_ptr_t)> _retrieve_chunk_callback = &_retrieve_chunk;
function<void (content_ptr_t)> _heartbeat_callback      = &_heartbeat;

at_key< int_<DEPOSIT_CHUNK>  >(callback_map) =  _deposit_chunk_callback;
at_key< int_<RETRIEVE_CHUNK> >(callback_map) =  _retrieve_chunk_callback;
at_key< int_<HEARTBEAT>      >(callback_map) =  _heartbeat_callback;

uledfs::intercom::masternode client(io_service, callback_map);
client.handle_init(hostname, DEPOSIT_CHUNK);

} // namespace
\end{lstlisting}
\item Every typedef name must have the \_t extenstion:

\begin{lstlisting}
typedef std::pair<uledfs::intercom::masternode, uledfs::intercom::ip> masternode_ip_pair_t
\end{lstlisting}

Also consult Working Draft, Standard for Programming Language C++ for more information. 

\end{itemize}

\section{Build}

\section{FUSE}

Please consult \href{http://www.gnu.org/software/libc/manual/html_node/index.html}{GNU C Library} (chapter 14 and 13) for more information regarding FUSE API.


\end{document}
